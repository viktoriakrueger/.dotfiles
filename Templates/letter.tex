% Briefvorlage für Privatleute
% Ersteller: Alexey Abel
% Git-Repository: https://github.com/PanCakeConnaisseur/latex-briefvorlage-din-5008
% Basiert auf KOMA-Scripts scrlttr2

\documentclass[
	% Schriftgröße
	fontsize=12pt,
	% zwischen Absätzen eine leere Zeile einfügen, statt lediglich Einrückung
	parskip=full,
	% Papierformat auf DIN-A4
	paper=A4,
	% Briefkopf (ganz oben) rechts ausrichten, standardmäßig links
	fromalign=right,
	% Telefonnummer im Briefkopf anzeigen
	fromphone=true,
	% Faxnnummer im Briefkopf anzeigen
	%fromfax=true,
	% E-Mail-Adresse im Briefkopf anzeigen
	fromemail=true,
	% URL im Briefkopf anzeigen
	%fromurl=true,
	% Faltmarkierungen verbergen
	foldmarks=false,
	% Die neuste Version von scrlettr2 verwenden
	version=last,
]{scrlttr2}

% Zeichenkodierung des Dokuments ist in UTF-8
\usepackage[utf8]{inputenc}

\usepackage[scaled]{helvet}
\renewcommand\familydefault{\sfdefault}

% Eurosymbol-Unterstützung
\usepackage{eurosym}
% Das Unicode-Zeichen € als \euro interpretieren.
% So kann man direkt € tippen anstatt jedes Mal \euro auszuschreiben.
\DeclareUnicodeCharacter{20AC}{\euro}

% Sprache des Dokuments auf Deutsch
\usepackage[english,ngerman]{babel}

% Includen von PDFs nach dem Brief, siehe \includepdf unten
\usepackage{pdfpages}

% klickbare Links und E-Mail-Adressen. Paket url kann keine klickbaren,
% deswegen hyperref. Option hidelinks versteckt farbigen Rahmen.
\usepackage[hidelinks]{hyperref}

\begin{document}

\pagestyle{empty}
\makeatletter
\@setplength{sigbeforevskip}{3em}
\makeatother

% Name nach Schlussgruß (unter Unterschrift) nicht nach rechts einrücken
\renewcommand*{\raggedsignature}{\raggedright}

% Absendername
\setkomavar{fromname}{Aleksej Chaichan}

% Absenderadresse
\setkomavar{fromaddress}{Kirchweg 80\\34119 Kassel}

% Absendertelefonnummer
\setkomavar{fromphone}{+49 172 683 0875}

% Absenderfax
% (oben fromfax=true setzen)
%\setkomavar{fromfax}{+49 222 222 22}

% Absender-E-Mail-Adresse
% der erste Paremeter ist fürs Klicken, der zweite wird angezeigt/gedruckt
\setkomavar{fromemail}{\href{mailto:aleksejchaichan@posteo.de}{aleksejchaichan@posteo.de}}

% Absender-URL
% (oben fromurl=true setzen)
% eckige Klammern entfernen damit "URL:" erscheint oder dort Alternativtext eintragen
% der erste Parameter ist fürs Klicken, der zweite wird angezeigt/gedruckt
\setkomavar{fromurl}[]{\href{http://absender.de}{absender.de}}

% Ort beim Datum
\setkomavar{place}{Kassel}

% Datum
\setkomavar{date}{\today}

% Betreff
\setkomavar{subject}{TITLE}

% Kundennummer
%\setkomavar{customer}[\customername]{DE-112233}

% Ihr Zeichen
%\setkomavar{yourref}[\yourrefname]{IZ-12345}

% Ihr Schreiben vom
%\setkomavar{yourmail}[\yourmailname]{1. April 2018}

\begin{letter}{
	ANREDE \\
	NAME \\
	ADRESSE NR \\
	PLZ ORT
}

\opening{Sehr geehrte...,}

BODY

\closing{ENDE}

% Post Scriptum
%\ps PS: Ich bin bis März nur telefonisch erreichbar.

% Anlage(n)
% Standardmäßig wird "Anlage(n)" eingefügt, dies kann überschrieben werden, hier mit "Anlagen"
%\setkomavar*{enclseparator}{Anlagen}
%\encl{Kopie des Ausweises}

% Verteiler
%\cc{Bürgermeister, Vereinsvorsitzender}

\end{letter}

% Weitere PDFs können automatisch angefügt werden, z.B. Ahnänge.
%\includepdf[pages=-,openright]{main.pdf}
% Pfad ist relativ zu dieser tex-Datei. Mit .. ein Verzeichnis hoch.
% Der pages-Parameter spezifiziert welche Seiten eingefügt werden.
% Beispiele:
% pages=-				alle Seiten
% pages={1-4}			Seite 1-4
% pages={1,4,5}			Seite 1, 4 und 5
% pages={3,{},8-11,15}	Seite 3, leere Seite, Seite 8-11 und Seite 15
% Der openright-Parameter startet die Anlagen auf ungerader (rechter) Seite, d.h. notfalls wird eine leere Seite
% eingefügt. Im doppelseitigem Druck wird dadurch besser zwischen Brief und Anlage getrennt. Für einseitigen Druck
% entfernen.

\end{document}
